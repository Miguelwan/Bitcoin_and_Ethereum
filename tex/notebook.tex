
% Default to the notebook output style

    


% Inherit from the specified cell style.




    
\documentclass[11pt]{article}

    
    
    \usepackage[T1]{fontenc}
    % Nicer default font (+ math font) than Computer Modern for most use cases
    \usepackage{mathpazo}

    % Basic figure setup, for now with no caption control since it's done
    % automatically by Pandoc (which extracts ![](path) syntax from Markdown).
    \usepackage{graphicx}
    % We will generate all images so they have a width \maxwidth. This means
    % that they will get their normal width if they fit onto the page, but
    % are scaled down if they would overflow the margins.
    \makeatletter
    \def\maxwidth{\ifdim\Gin@nat@width>\linewidth\linewidth
    \else\Gin@nat@width\fi}
    \makeatother
    \let\Oldincludegraphics\includegraphics
    % Set max figure width to be 80% of text width, for now hardcoded.
    \renewcommand{\includegraphics}[1]{\Oldincludegraphics[width=.8\maxwidth]{#1}}
    % Ensure that by default, figures have no caption (until we provide a
    % proper Figure object with a Caption API and a way to capture that
    % in the conversion process - todo).
    \usepackage{caption}
    \DeclareCaptionLabelFormat{nolabel}{}
    \captionsetup{labelformat=nolabel}

    \usepackage{adjustbox} % Used to constrain images to a maximum size 
    \usepackage{xcolor} % Allow colors to be defined
    \usepackage{enumerate} % Needed for markdown enumerations to work
    \usepackage{geometry} % Used to adjust the document margins
    \usepackage{amsmath} % Equations
    \usepackage{amssymb} % Equations
    \usepackage{textcomp} % defines textquotesingle
    % Hack from http://tex.stackexchange.com/a/47451/13684:
    \AtBeginDocument{%
        \def\PYZsq{\textquotesingle}% Upright quotes in Pygmentized code
    }
    \usepackage{upquote} % Upright quotes for verbatim code
    \usepackage{eurosym} % defines \euro
    \usepackage[mathletters]{ucs} % Extended unicode (utf-8) support
    \usepackage[utf8x]{inputenc} % Allow utf-8 characters in the tex document
    \usepackage{fancyvrb} % verbatim replacement that allows latex
    \usepackage{grffile} % extends the file name processing of package graphics 
                         % to support a larger range 
    % The hyperref package gives us a pdf with properly built
    % internal navigation ('pdf bookmarks' for the table of contents,
    % internal cross-reference links, web links for URLs, etc.)
    \usepackage{hyperref}
    \usepackage{longtable} % longtable support required by pandoc >1.10
    \usepackage{booktabs}  % table support for pandoc > 1.12.2
    \usepackage[inline]{enumitem} % IRkernel/repr support (it uses the enumerate* environment)
    \usepackage[normalem]{ulem} % ulem is needed to support strikethroughs (\sout)
                                % normalem makes italics be italics, not underlines
    

    
    
    % Colors for the hyperref package
    \definecolor{urlcolor}{rgb}{0,.145,.698}
    \definecolor{linkcolor}{rgb}{.71,0.21,0.01}
    \definecolor{citecolor}{rgb}{.12,.54,.11}

    % ANSI colors
    \definecolor{ansi-black}{HTML}{3E424D}
    \definecolor{ansi-black-intense}{HTML}{282C36}
    \definecolor{ansi-red}{HTML}{E75C58}
    \definecolor{ansi-red-intense}{HTML}{B22B31}
    \definecolor{ansi-green}{HTML}{00A250}
    \definecolor{ansi-green-intense}{HTML}{007427}
    \definecolor{ansi-yellow}{HTML}{DDB62B}
    \definecolor{ansi-yellow-intense}{HTML}{B27D12}
    \definecolor{ansi-blue}{HTML}{208FFB}
    \definecolor{ansi-blue-intense}{HTML}{0065CA}
    \definecolor{ansi-magenta}{HTML}{D160C4}
    \definecolor{ansi-magenta-intense}{HTML}{A03196}
    \definecolor{ansi-cyan}{HTML}{60C6C8}
    \definecolor{ansi-cyan-intense}{HTML}{258F8F}
    \definecolor{ansi-white}{HTML}{C5C1B4}
    \definecolor{ansi-white-intense}{HTML}{A1A6B2}

    % commands and environments needed by pandoc snippets
    % extracted from the output of `pandoc -s`
    \providecommand{\tightlist}{%
      \setlength{\itemsep}{0pt}\setlength{\parskip}{0pt}}
    \DefineVerbatimEnvironment{Highlighting}{Verbatim}{commandchars=\\\{\}}
    % Add ',fontsize=\small' for more characters per line
    \newenvironment{Shaded}{}{}
    \newcommand{\KeywordTok}[1]{\textcolor[rgb]{0.00,0.44,0.13}{\textbf{{#1}}}}
    \newcommand{\DataTypeTok}[1]{\textcolor[rgb]{0.56,0.13,0.00}{{#1}}}
    \newcommand{\DecValTok}[1]{\textcolor[rgb]{0.25,0.63,0.44}{{#1}}}
    \newcommand{\BaseNTok}[1]{\textcolor[rgb]{0.25,0.63,0.44}{{#1}}}
    \newcommand{\FloatTok}[1]{\textcolor[rgb]{0.25,0.63,0.44}{{#1}}}
    \newcommand{\CharTok}[1]{\textcolor[rgb]{0.25,0.44,0.63}{{#1}}}
    \newcommand{\StringTok}[1]{\textcolor[rgb]{0.25,0.44,0.63}{{#1}}}
    \newcommand{\CommentTok}[1]{\textcolor[rgb]{0.38,0.63,0.69}{\textit{{#1}}}}
    \newcommand{\OtherTok}[1]{\textcolor[rgb]{0.00,0.44,0.13}{{#1}}}
    \newcommand{\AlertTok}[1]{\textcolor[rgb]{1.00,0.00,0.00}{\textbf{{#1}}}}
    \newcommand{\FunctionTok}[1]{\textcolor[rgb]{0.02,0.16,0.49}{{#1}}}
    \newcommand{\RegionMarkerTok}[1]{{#1}}
    \newcommand{\ErrorTok}[1]{\textcolor[rgb]{1.00,0.00,0.00}{\textbf{{#1}}}}
    \newcommand{\NormalTok}[1]{{#1}}
    
    % Additional commands for more recent versions of Pandoc
    \newcommand{\ConstantTok}[1]{\textcolor[rgb]{0.53,0.00,0.00}{{#1}}}
    \newcommand{\SpecialCharTok}[1]{\textcolor[rgb]{0.25,0.44,0.63}{{#1}}}
    \newcommand{\VerbatimStringTok}[1]{\textcolor[rgb]{0.25,0.44,0.63}{{#1}}}
    \newcommand{\SpecialStringTok}[1]{\textcolor[rgb]{0.73,0.40,0.53}{{#1}}}
    \newcommand{\ImportTok}[1]{{#1}}
    \newcommand{\DocumentationTok}[1]{\textcolor[rgb]{0.73,0.13,0.13}{\textit{{#1}}}}
    \newcommand{\AnnotationTok}[1]{\textcolor[rgb]{0.38,0.63,0.69}{\textbf{\textit{{#1}}}}}
    \newcommand{\CommentVarTok}[1]{\textcolor[rgb]{0.38,0.63,0.69}{\textbf{\textit{{#1}}}}}
    \newcommand{\VariableTok}[1]{\textcolor[rgb]{0.10,0.09,0.49}{{#1}}}
    \newcommand{\ControlFlowTok}[1]{\textcolor[rgb]{0.00,0.44,0.13}{\textbf{{#1}}}}
    \newcommand{\OperatorTok}[1]{\textcolor[rgb]{0.40,0.40,0.40}{{#1}}}
    \newcommand{\BuiltInTok}[1]{{#1}}
    \newcommand{\ExtensionTok}[1]{{#1}}
    \newcommand{\PreprocessorTok}[1]{\textcolor[rgb]{0.74,0.48,0.00}{{#1}}}
    \newcommand{\AttributeTok}[1]{\textcolor[rgb]{0.49,0.56,0.16}{{#1}}}
    \newcommand{\InformationTok}[1]{\textcolor[rgb]{0.38,0.63,0.69}{\textbf{\textit{{#1}}}}}
    \newcommand{\WarningTok}[1]{\textcolor[rgb]{0.38,0.63,0.69}{\textbf{\textit{{#1}}}}}
    
    
    % Define a nice break command that doesn't care if a line doesn't already
    % exist.
    \def\br{\hspace*{\fill} \\* }
    % Math Jax compatability definitions
    \def\gt{>}
    \def\lt{<}
    % Document parameters
    \title{Bitcoin vs Ethereum}
    \author{Miguel Moreno}
    
    
    

    % Pygments definitions
    
\makeatletter
\def\PY@reset{\let\PY@it=\relax \let\PY@bf=\relax%
    \let\PY@ul=\relax \let\PY@tc=\relax%
    \let\PY@bc=\relax \let\PY@ff=\relax}
\def\PY@tok#1{\csname PY@tok@#1\endcsname}
\def\PY@toks#1+{\ifx\relax#1\empty\else%
    \PY@tok{#1}\expandafter\PY@toks\fi}
\def\PY@do#1{\PY@bc{\PY@tc{\PY@ul{%
    \PY@it{\PY@bf{\PY@ff{#1}}}}}}}
\def\PY#1#2{\PY@reset\PY@toks#1+\relax+\PY@do{#2}}

\expandafter\def\csname PY@tok@w\endcsname{\def\PY@tc##1{\textcolor[rgb]{0.73,0.73,0.73}{##1}}}
\expandafter\def\csname PY@tok@c\endcsname{\let\PY@it=\textit\def\PY@tc##1{\textcolor[rgb]{0.25,0.50,0.50}{##1}}}
\expandafter\def\csname PY@tok@cp\endcsname{\def\PY@tc##1{\textcolor[rgb]{0.74,0.48,0.00}{##1}}}
\expandafter\def\csname PY@tok@k\endcsname{\let\PY@bf=\textbf\def\PY@tc##1{\textcolor[rgb]{0.00,0.50,0.00}{##1}}}
\expandafter\def\csname PY@tok@kp\endcsname{\def\PY@tc##1{\textcolor[rgb]{0.00,0.50,0.00}{##1}}}
\expandafter\def\csname PY@tok@kt\endcsname{\def\PY@tc##1{\textcolor[rgb]{0.69,0.00,0.25}{##1}}}
\expandafter\def\csname PY@tok@o\endcsname{\def\PY@tc##1{\textcolor[rgb]{0.40,0.40,0.40}{##1}}}
\expandafter\def\csname PY@tok@ow\endcsname{\let\PY@bf=\textbf\def\PY@tc##1{\textcolor[rgb]{0.67,0.13,1.00}{##1}}}
\expandafter\def\csname PY@tok@nb\endcsname{\def\PY@tc##1{\textcolor[rgb]{0.00,0.50,0.00}{##1}}}
\expandafter\def\csname PY@tok@nf\endcsname{\def\PY@tc##1{\textcolor[rgb]{0.00,0.00,1.00}{##1}}}
\expandafter\def\csname PY@tok@nc\endcsname{\let\PY@bf=\textbf\def\PY@tc##1{\textcolor[rgb]{0.00,0.00,1.00}{##1}}}
\expandafter\def\csname PY@tok@nn\endcsname{\let\PY@bf=\textbf\def\PY@tc##1{\textcolor[rgb]{0.00,0.00,1.00}{##1}}}
\expandafter\def\csname PY@tok@ne\endcsname{\let\PY@bf=\textbf\def\PY@tc##1{\textcolor[rgb]{0.82,0.25,0.23}{##1}}}
\expandafter\def\csname PY@tok@nv\endcsname{\def\PY@tc##1{\textcolor[rgb]{0.10,0.09,0.49}{##1}}}
\expandafter\def\csname PY@tok@no\endcsname{\def\PY@tc##1{\textcolor[rgb]{0.53,0.00,0.00}{##1}}}
\expandafter\def\csname PY@tok@nl\endcsname{\def\PY@tc##1{\textcolor[rgb]{0.63,0.63,0.00}{##1}}}
\expandafter\def\csname PY@tok@ni\endcsname{\let\PY@bf=\textbf\def\PY@tc##1{\textcolor[rgb]{0.60,0.60,0.60}{##1}}}
\expandafter\def\csname PY@tok@na\endcsname{\def\PY@tc##1{\textcolor[rgb]{0.49,0.56,0.16}{##1}}}
\expandafter\def\csname PY@tok@nt\endcsname{\let\PY@bf=\textbf\def\PY@tc##1{\textcolor[rgb]{0.00,0.50,0.00}{##1}}}
\expandafter\def\csname PY@tok@nd\endcsname{\def\PY@tc##1{\textcolor[rgb]{0.67,0.13,1.00}{##1}}}
\expandafter\def\csname PY@tok@s\endcsname{\def\PY@tc##1{\textcolor[rgb]{0.73,0.13,0.13}{##1}}}
\expandafter\def\csname PY@tok@sd\endcsname{\let\PY@it=\textit\def\PY@tc##1{\textcolor[rgb]{0.73,0.13,0.13}{##1}}}
\expandafter\def\csname PY@tok@si\endcsname{\let\PY@bf=\textbf\def\PY@tc##1{\textcolor[rgb]{0.73,0.40,0.53}{##1}}}
\expandafter\def\csname PY@tok@se\endcsname{\let\PY@bf=\textbf\def\PY@tc##1{\textcolor[rgb]{0.73,0.40,0.13}{##1}}}
\expandafter\def\csname PY@tok@sr\endcsname{\def\PY@tc##1{\textcolor[rgb]{0.73,0.40,0.53}{##1}}}
\expandafter\def\csname PY@tok@ss\endcsname{\def\PY@tc##1{\textcolor[rgb]{0.10,0.09,0.49}{##1}}}
\expandafter\def\csname PY@tok@sx\endcsname{\def\PY@tc##1{\textcolor[rgb]{0.00,0.50,0.00}{##1}}}
\expandafter\def\csname PY@tok@m\endcsname{\def\PY@tc##1{\textcolor[rgb]{0.40,0.40,0.40}{##1}}}
\expandafter\def\csname PY@tok@gh\endcsname{\let\PY@bf=\textbf\def\PY@tc##1{\textcolor[rgb]{0.00,0.00,0.50}{##1}}}
\expandafter\def\csname PY@tok@gu\endcsname{\let\PY@bf=\textbf\def\PY@tc##1{\textcolor[rgb]{0.50,0.00,0.50}{##1}}}
\expandafter\def\csname PY@tok@gd\endcsname{\def\PY@tc##1{\textcolor[rgb]{0.63,0.00,0.00}{##1}}}
\expandafter\def\csname PY@tok@gi\endcsname{\def\PY@tc##1{\textcolor[rgb]{0.00,0.63,0.00}{##1}}}
\expandafter\def\csname PY@tok@gr\endcsname{\def\PY@tc##1{\textcolor[rgb]{1.00,0.00,0.00}{##1}}}
\expandafter\def\csname PY@tok@ge\endcsname{\let\PY@it=\textit}
\expandafter\def\csname PY@tok@gs\endcsname{\let\PY@bf=\textbf}
\expandafter\def\csname PY@tok@gp\endcsname{\let\PY@bf=\textbf\def\PY@tc##1{\textcolor[rgb]{0.00,0.00,0.50}{##1}}}
\expandafter\def\csname PY@tok@go\endcsname{\def\PY@tc##1{\textcolor[rgb]{0.53,0.53,0.53}{##1}}}
\expandafter\def\csname PY@tok@gt\endcsname{\def\PY@tc##1{\textcolor[rgb]{0.00,0.27,0.87}{##1}}}
\expandafter\def\csname PY@tok@err\endcsname{\def\PY@bc##1{\setlength{\fboxsep}{0pt}\fcolorbox[rgb]{1.00,0.00,0.00}{1,1,1}{\strut ##1}}}
\expandafter\def\csname PY@tok@kc\endcsname{\let\PY@bf=\textbf\def\PY@tc##1{\textcolor[rgb]{0.00,0.50,0.00}{##1}}}
\expandafter\def\csname PY@tok@kd\endcsname{\let\PY@bf=\textbf\def\PY@tc##1{\textcolor[rgb]{0.00,0.50,0.00}{##1}}}
\expandafter\def\csname PY@tok@kn\endcsname{\let\PY@bf=\textbf\def\PY@tc##1{\textcolor[rgb]{0.00,0.50,0.00}{##1}}}
\expandafter\def\csname PY@tok@kr\endcsname{\let\PY@bf=\textbf\def\PY@tc##1{\textcolor[rgb]{0.00,0.50,0.00}{##1}}}
\expandafter\def\csname PY@tok@bp\endcsname{\def\PY@tc##1{\textcolor[rgb]{0.00,0.50,0.00}{##1}}}
\expandafter\def\csname PY@tok@fm\endcsname{\def\PY@tc##1{\textcolor[rgb]{0.00,0.00,1.00}{##1}}}
\expandafter\def\csname PY@tok@vc\endcsname{\def\PY@tc##1{\textcolor[rgb]{0.10,0.09,0.49}{##1}}}
\expandafter\def\csname PY@tok@vg\endcsname{\def\PY@tc##1{\textcolor[rgb]{0.10,0.09,0.49}{##1}}}
\expandafter\def\csname PY@tok@vi\endcsname{\def\PY@tc##1{\textcolor[rgb]{0.10,0.09,0.49}{##1}}}
\expandafter\def\csname PY@tok@vm\endcsname{\def\PY@tc##1{\textcolor[rgb]{0.10,0.09,0.49}{##1}}}
\expandafter\def\csname PY@tok@sa\endcsname{\def\PY@tc##1{\textcolor[rgb]{0.73,0.13,0.13}{##1}}}
\expandafter\def\csname PY@tok@sb\endcsname{\def\PY@tc##1{\textcolor[rgb]{0.73,0.13,0.13}{##1}}}
\expandafter\def\csname PY@tok@sc\endcsname{\def\PY@tc##1{\textcolor[rgb]{0.73,0.13,0.13}{##1}}}
\expandafter\def\csname PY@tok@dl\endcsname{\def\PY@tc##1{\textcolor[rgb]{0.73,0.13,0.13}{##1}}}
\expandafter\def\csname PY@tok@s2\endcsname{\def\PY@tc##1{\textcolor[rgb]{0.73,0.13,0.13}{##1}}}
\expandafter\def\csname PY@tok@sh\endcsname{\def\PY@tc##1{\textcolor[rgb]{0.73,0.13,0.13}{##1}}}
\expandafter\def\csname PY@tok@s1\endcsname{\def\PY@tc##1{\textcolor[rgb]{0.73,0.13,0.13}{##1}}}
\expandafter\def\csname PY@tok@mb\endcsname{\def\PY@tc##1{\textcolor[rgb]{0.40,0.40,0.40}{##1}}}
\expandafter\def\csname PY@tok@mf\endcsname{\def\PY@tc##1{\textcolor[rgb]{0.40,0.40,0.40}{##1}}}
\expandafter\def\csname PY@tok@mh\endcsname{\def\PY@tc##1{\textcolor[rgb]{0.40,0.40,0.40}{##1}}}
\expandafter\def\csname PY@tok@mi\endcsname{\def\PY@tc##1{\textcolor[rgb]{0.40,0.40,0.40}{##1}}}
\expandafter\def\csname PY@tok@il\endcsname{\def\PY@tc##1{\textcolor[rgb]{0.40,0.40,0.40}{##1}}}
\expandafter\def\csname PY@tok@mo\endcsname{\def\PY@tc##1{\textcolor[rgb]{0.40,0.40,0.40}{##1}}}
\expandafter\def\csname PY@tok@ch\endcsname{\let\PY@it=\textit\def\PY@tc##1{\textcolor[rgb]{0.25,0.50,0.50}{##1}}}
\expandafter\def\csname PY@tok@cm\endcsname{\let\PY@it=\textit\def\PY@tc##1{\textcolor[rgb]{0.25,0.50,0.50}{##1}}}
\expandafter\def\csname PY@tok@cpf\endcsname{\let\PY@it=\textit\def\PY@tc##1{\textcolor[rgb]{0.25,0.50,0.50}{##1}}}
\expandafter\def\csname PY@tok@c1\endcsname{\let\PY@it=\textit\def\PY@tc##1{\textcolor[rgb]{0.25,0.50,0.50}{##1}}}
\expandafter\def\csname PY@tok@cs\endcsname{\let\PY@it=\textit\def\PY@tc##1{\textcolor[rgb]{0.25,0.50,0.50}{##1}}}

\def\PYZbs{\char`\\}
\def\PYZus{\char`\_}
\def\PYZob{\char`\{}
\def\PYZcb{\char`\}}
\def\PYZca{\char`\^}
\def\PYZam{\char`\&}
\def\PYZlt{\char`\<}
\def\PYZgt{\char`\>}
\def\PYZsh{\char`\#}
\def\PYZpc{\char`\%}
\def\PYZdl{\char`\$}
\def\PYZhy{\char`\-}
\def\PYZsq{\char`\'}
\def\PYZdq{\char`\"}
\def\PYZti{\char`\~}
% for compatibility with earlier versions
\def\PYZat{@}
\def\PYZlb{[}
\def\PYZrb{]}
\makeatother


    % Exact colors from NB
    \definecolor{incolor}{rgb}{0.0, 0.0, 0.5}
    \definecolor{outcolor}{rgb}{0.545, 0.0, 0.0}



    
    % Prevent overflowing lines due to hard-to-break entities
    \sloppy 
    % Setup hyperref package
    \hypersetup{
      breaklinks=true,  % so long urls are correctly broken across lines
      colorlinks=true,
      urlcolor=urlcolor,
      linkcolor=linkcolor,
      citecolor=citecolor,
      }
    % Slightly bigger margins than the latex defaults
    
    \geometry{verbose,tmargin=1in,bmargin=1in,lmargin=1in,rmargin=1in}
    
    

    \begin{document}
    
    
    \maketitle
    
    

   \begin{abstract}
   In this notebook we study the correlation between the values
of Bitcoin and Ethereum during the year 2020, by market capitalization.
We found that the values of 2020 resembles a linear relation. We model
this relation by a linear regression and test this model with the value
of this two cryptoassets of January 2021.
   \end{abstract}

%Abstract: 

    \section{Introduction}\label{introduction}

During the last years we have seen a raise on the popularity of Bitcoin
and, as a consequence, a raise on the value of Bitcoin, in July 2020
Bitcoin reached a value over the 10000 USD. Due to its quick grow many
people see Bitcoin as a booble and as an unstable assets.

Many articles and news have been written about the high value of
Bitcoin, but few were written about some other cryptoassets. Bitcoin is
not the only cryptoassets with a high popularity and gorwing rate.
Ethereum is the second most popular cryptocurrency which has also gained
a lot of popularity in the recent years, this has been unnoticed by most
of the people and mainstream news. It is natural to ask whether there is
a relation between the values of the different cryptocurrencies, in this
notebook we study this question for Bitcoin and Ethereum during the
2020.

Cryptocurrencies are not the only cryptoassets, the introduction of
Blockchain 2.0 allowed the introduction of different kinds of Token
which uses blockchains that already exist, such as the Basic Attention
Token from the browser Brave which usses the Ethereum blockchain. It is
easy to jump to the conclusion that a rise in the popularity of Ethereum
would imply a rise on the popularity of tokens based on the Ethereum
blockchain. In this paper we study the question whether an increase on
the popularity of Ethereum implies an increase on the popularity of the
Basic Attention Token.

This notebook and the code used in it can be found in the github of the
author. The pdf version of this notebook can be found in the webpage of
the author.

    \section{Preliminaries}\label{preliminaries}

This is a notebook running in Googlecolab in which we will be using R
for coding and we will use the data from Yahoo finance. Therefore we
have to activate the magic for R and obtain tha data from Yahoo. In this
section we will activate R and install/activate the packages that we
will use.

    Activation of R magic

    \begin{Verbatim}[commandchars=\\\{\}]
{\color{incolor}In [{\color{incolor} }]:} \PY{o}{\PYZpc{}}\PY{k}{load\PYZus{}ext} rpy2.ipython
\end{Verbatim}


    \begin{Verbatim}[commandchars=\\\{\}]
/usr/local/lib/python3.6/dist-packages/rpy2/robjects/pandas2ri.py:14: FutureWarning: pandas.core.index is deprecated and will be removed in a future version.  The public classes are available in the top-level namespace.
  from pandas.core.index import Index as PandasIndex
/usr/local/lib/python3.6/dist-packages/rpy2/robjects/pandas2ri.py:34: UserWarning: pandas >= 1.0 is not supported.
  warnings.warn('pandas >= 1.0 is not supported.')

    \end{Verbatim}

    Intall and load the packages

    \begin{Verbatim}[commandchars=\\\{\}]
{\color{incolor}In [{\color{incolor} }]:} \PY{o}{\PYZpc{}\PYZpc{}}R
        install.packages\PY{p}{(}\PY{l+s}{\PYZsq{}}\PY{l+s}{quantmod\PYZsq{}}\PY{p}{)}
        install.packages\PY{p}{(}\PY{l+s}{\PYZsq{}}\PY{l+s}{dplyr\PYZsq{}}\PY{p}{)}
        install.packages\PY{p}{(}\PY{l+s}{\PYZsq{}}\PY{l+s}{tidyverse\PYZsq{}}\PY{p}{)}
        install.packages\PY{p}{(}\PY{l+s}{\PYZsq{}}\PY{l+s}{ggplot2\PYZsq{}}\PY{p}{)}
        install.packages\PY{p}{(}\PY{l+s}{\PYZsq{}}\PY{l+s}{caTools\PYZsq{}}\PY{p}{)}
        \PY{k+kn}{library}\PY{p}{(}quantmod\PY{p}{)}
        \PY{k+kn}{library}\PY{p}{(}dplyr\PY{p}{)}
        \PY{k+kn}{library}\PY{p}{(}tidyverse\PY{p}{)}
        \PY{k+kn}{library}\PY{p}{(}ggplot2\PY{p}{)}
        \PY{k+kn}{library}\PY{p}{(}caTools\PY{p}{)}
\end{Verbatim}


    \section{Preparation and exploration of the
data}\label{preparation-and-exploration-of-the-data}

In this section we proceed to obtain the data and do a basic exploration
of it, to get a basic overview of the values. We will use the data of
Bitcoin, Ethereum, and the Basic Attention Token. For each one we will
download the data an plot the Japanese Candllesticks. We will work with
the adjusted value.

    We will study the time interval os 2020, from 1st of January to the 31st
of December. We have to define the interval so we can proced with the
downloading of the data.

    \begin{Verbatim}[commandchars=\\\{\}]
{\color{incolor}In [{\color{incolor} }]:} \PY{o}{\PYZpc{}\PYZpc{}}R
        starting\PYZus{}date \PY{o}{\PYZlt{}\PYZhy{}} \PY{l+s}{\PYZdq{}}\PY{l+s}{2020\PYZhy{}1\PYZhy{}1\PYZdq{}}
        final\PYZus{}date \PY{o}{\PYZlt{}\PYZhy{}} \PY{l+s}{\PYZdq{}}\PY{l+s}{2020\PYZhy{}12\PYZhy{}31\PYZdq{}}
\end{Verbatim}


    \subsection{Bitcoin}\label{bitcoin}

    \begin{Verbatim}[commandchars=\\\{\}]
{\color{incolor}In [{\color{incolor} }]:} \PY{o}{\PYZpc{}\PYZpc{}}R
        BTCUSD \PY{o}{\PYZlt{}\PYZhy{}} getSymbols.yahoo\PY{p}{(}\PY{l+s}{\PYZdq{}}\PY{l+s}{BTC\PYZhy{}USD\PYZdq{}}\PY{p}{,} from \PY{o}{=} starting\PYZus{}date\PY{p}{,} to \PY{o}{=} final\PYZus{}date\PY{p}{,} auto.assign \PY{o}{=} \PY{n+nb+bp}{F}\PY{p}{)}\PY{p}{[}\PY{p}{,}\PY{p}{]}
        BTCAdjust \PY{o}{\PYZlt{}\PYZhy{}} BTCUSD\PY{p}{[}\PY{p}{,}\PY{l+m}{6}\PY{p}{]}
        chartSeries\PY{p}{(}BTCUSD\PY{p}{)}
\end{Verbatim}


    \begin{center}
    \adjustimage{max size={0.9\linewidth}{0.9\paperheight}}{output_11_0.png}
    \end{center}
    { \hspace*{\fill} \\}
    
    \subsection{Ethereum}\label{ethereum}

    \begin{Verbatim}[commandchars=\\\{\}]
{\color{incolor}In [{\color{incolor} }]:} \PY{o}{\PYZpc{}\PYZpc{}}R
        ETHUSD \PY{o}{\PYZlt{}\PYZhy{}} getSymbols.yahoo\PY{p}{(}\PY{l+s}{\PYZdq{}}\PY{l+s}{ETH\PYZhy{}USD\PYZdq{}}\PY{p}{,} from \PY{o}{=} starting\PYZus{}date\PY{p}{,} to \PY{o}{=} final\PYZus{}date\PY{p}{,} auto.assign \PY{o}{=} \PY{n+nb+bp}{F}\PY{p}{)}\PY{p}{[}\PY{p}{,}\PY{p}{]}
        ETHAdjust \PY{o}{\PYZlt{}\PYZhy{}} ETHUSD\PY{p}{[}\PY{p}{,}\PY{l+m}{6}\PY{p}{]}
        chartSeries\PY{p}{(}ETHUSD\PY{p}{)}
\end{Verbatim}


    \begin{center}
    \adjustimage{max size={0.9\linewidth}{0.9\paperheight}}{output_13_0.png}
    \end{center}
    { \hspace*{\fill} \\}
    
    \subsection{Basic Attention Token}\label{basic-attention-token}

    \begin{Verbatim}[commandchars=\\\{\}]
{\color{incolor}In [{\color{incolor} }]:} \PY{o}{\PYZpc{}\PYZpc{}}R
        BATUSD \PY{o}{\PYZlt{}\PYZhy{}} getSymbols.yahoo\PY{p}{(}\PY{l+s}{\PYZdq{}}\PY{l+s}{BAT\PYZhy{}USD\PYZdq{}}\PY{p}{,} from \PY{o}{=} starting\PYZus{}date\PY{p}{,} to \PY{o}{=} final\PYZus{}date\PY{p}{,} auto.assign \PY{o}{=} \PY{n+nb+bp}{F}\PY{p}{)}\PY{p}{[}\PY{p}{,}\PY{p}{]}
        BATAdjust \PY{o}{\PYZlt{}\PYZhy{}} BATUSD\PY{p}{[}\PY{p}{,}\PY{l+m}{6}\PY{p}{]}
        chartSeries\PY{p}{(}BATUSD\PY{p}{)}
\end{Verbatim}


    \begin{center}
    \adjustimage{max size={0.9\linewidth}{0.9\paperheight}}{output_15_0.png}
    \end{center}
    { \hspace*{\fill} \\}
    
    \subsection{Exploration of Bitcoin vs
Ethereum}\label{exploration-of-bitcoin-vs-ethereum}

As we could see in the previous images, Bitcoin and Ethereum have
similarities in the behaviour of their value. The most notorious are the
months of February, October, and September. To explore this similarities
we should look at different indeces.

    Let us start by taking a close look to the adjustment, below we find the
plot of the adjustment, on top is the Bitcoin and Ethereum is below.

    \begin{Verbatim}[commandchars=\\\{\}]
{\color{incolor}In [{\color{incolor} }]:} \PY{o}{\PYZpc{}\PYZpc{}}R
        chartSeries\PY{p}{(}BTCAdjust\PY{p}{)}
        chartSeries\PY{p}{(}ETHAdjust\PY{p}{)}
\end{Verbatim}


    \begin{center}
    \adjustimage{max size={0.9\linewidth}{0.9\paperheight}}{output_18_0.png}
    \end{center}
    { \hspace*{\fill} \\}
    
    \begin{center}
    \adjustimage{max size={0.9\linewidth}{0.9\paperheight}}{output_18_1.png}
    \end{center}
    { \hspace*{\fill} \\}
    
    Now lets take a look to the Moving Convergence Divergence (MACD), on top
we can find the plot of BItcoin and below is the Ethereum one.

    \begin{Verbatim}[commandchars=\\\{\}]
{\color{incolor}In [{\color{incolor} }]:} \PY{o}{\PYZpc{}\PYZpc{}}R
        \PY{l+s+sb}{`BTCUSD`} \PY{o}{\PYZpc{}\PYZgt{}\PYZpc{}} chartSeries\PY{p}{(}TA \PY{o}{=} \PY{l+s}{\PYZdq{}}\PY{l+s}{addMACD()\PYZdq{}}\PY{p}{)}
        \PY{l+s+sb}{`ETHUSD`} \PY{o}{\PYZpc{}\PYZgt{}\PYZpc{}} chartSeries\PY{p}{(}TA \PY{o}{=} \PY{l+s}{\PYZdq{}}\PY{l+s}{addMACD()\PYZdq{}}\PY{p}{)}
\end{Verbatim}


    \begin{center}
    \adjustimage{max size={0.9\linewidth}{0.9\paperheight}}{output_20_0.png}
    \end{center}
    { \hspace*{\fill} \\}
    
    \begin{center}
    \adjustimage{max size={0.9\linewidth}{0.9\paperheight}}{output_20_1.png}
    \end{center}
    { \hspace*{\fill} \\}
    
    Now lets take a look to the Relative Strength Index (RSI), on top we can
find the plot of BItcoin and below is the Ethereum one.

    \begin{Verbatim}[commandchars=\\\{\}]
{\color{incolor}In [{\color{incolor} }]:} \PY{o}{\PYZpc{}\PYZpc{}}R
        \PY{l+s+sb}{`BTCUSD`} \PY{o}{\PYZpc{}\PYZgt{}\PYZpc{}} chartSeries\PY{p}{(}TA \PY{o}{=} \PY{l+s}{\PYZdq{}}\PY{l+s}{addRSI()\PYZdq{}}\PY{p}{)}
        \PY{l+s+sb}{`ETHUSD`} \PY{o}{\PYZpc{}\PYZgt{}\PYZpc{}} chartSeries\PY{p}{(}TA \PY{o}{=} \PY{l+s}{\PYZdq{}}\PY{l+s}{addRSI()\PYZdq{}}\PY{p}{)}
\end{Verbatim}


    \begin{center}
    \adjustimage{max size={0.9\linewidth}{0.9\paperheight}}{output_22_0.png}
    \end{center}
    { \hspace*{\fill} \\}
    
    \begin{center}
    \adjustimage{max size={0.9\linewidth}{0.9\paperheight}}{output_22_1.png}
    \end{center}
    { \hspace*{\fill} \\}
    
    Now lets take a look to the Average True Range (ATR), on top we can find
the plot of BItcoin and below is the Ethereum one.

    \begin{Verbatim}[commandchars=\\\{\}]
{\color{incolor}In [{\color{incolor} }]:} \PY{o}{\PYZpc{}\PYZpc{}}R
        \PY{l+s+sb}{`BTCUSD`} \PY{o}{\PYZpc{}\PYZgt{}\PYZpc{}} chartSeries\PY{p}{(}TA \PY{o}{=} \PY{l+s}{\PYZdq{}}\PY{l+s}{addATR()\PYZdq{}}\PY{p}{)}
        \PY{l+s+sb}{`ETHUSD`} \PY{o}{\PYZpc{}\PYZgt{}\PYZpc{}} chartSeries\PY{p}{(}TA \PY{o}{=} \PY{l+s}{\PYZdq{}}\PY{l+s}{addATR()\PYZdq{}}\PY{p}{)}
\end{Verbatim}


    \begin{center}
    \adjustimage{max size={0.9\linewidth}{0.9\paperheight}}{output_24_0.png}
    \end{center}
    { \hspace*{\fill} \\}
    
    \begin{center}
    \adjustimage{max size={0.9\linewidth}{0.9\paperheight}}{output_24_1.png}
    \end{center}
    { \hspace*{\fill} \\}
    
    To get a better overview of the possible correlation between the value
of these two assets, we shoul plot the variation of adjustment of both
in the same graphic with respect of the date.

    \begin{Verbatim}[commandchars=\\\{\}]
{\color{incolor}In [{\color{incolor} }]:} \PY{o}{\PYZpc{}\PYZpc{}}R
        tickersEB \PY{o}{\PYZlt{}\PYZhy{}} \PY{k+kt}{c}\PY{p}{(}\PY{l+s}{\PYZdq{}}\PY{l+s}{ETH\PYZhy{}USD\PYZdq{}}\PY{p}{,} \PY{l+s}{\PYZdq{}}\PY{l+s}{BTC\PYZhy{}USD\PYZdq{}}\PY{p}{)}
        portfolioPricesEB \PY{o}{\PYZlt{}\PYZhy{}} \PY{k+kc}{NULL}
        \PY{k+kr}{for} \PY{p}{(}ticker \PY{k+kr}{in} tickersEB\PY{p}{)}\PY{p}{\PYZob{}}
          portfolioPricesEB \PY{o}{\PYZlt{}\PYZhy{}} \PY{k+kp}{cbind}\PY{p}{(}portfolioPricesEB\PY{p}{,} 
                                   getSymbols.yahoo\PY{p}{(}ticker\PY{p}{,}
                                                              from \PY{o}{=} starting\PYZus{}date\PY{p}{,} 
                                                              to \PY{o}{=} final\PYZus{}date\PY{p}{,}
                                                              periodicity \PY{o}{=} \PY{l+s}{\PYZdq{}}\PY{l+s}{daily\PYZdq{}}\PY{p}{,}
                                                              auto.assign \PY{o}{=} \PY{n+nb+bp}{F}\PY{p}{)}\PY{p}{[}\PY{p}{,}\PY{l+m}{6}\PY{p}{]}\PY{p}{)}
          
        \PY{p}{\PYZcb{}}
\end{Verbatim}


    \begin{Verbatim}[commandchars=\\\{\}]
{\color{incolor}In [{\color{incolor} }]:} \PY{o}{\PYZpc{}\PYZpc{}}R
        portfolioPricesEB \PY{o}{\PYZlt{}\PYZhy{}} \PY{k+kp}{as.data.frame}\PY{p}{(}portfolioPricesEB\PY{p}{)}
        portfolioPricesEB \PY{o}{\PYZlt{}\PYZhy{}} rownames\PYZus{}to\PYZus{}column\PY{p}{(}portfolioPricesEB\PY{p}{,} var \PY{o}{=} \PY{l+s}{\PYZdq{}}\PY{l+s}{Date\PYZdq{}}\PY{p}{)}
        dfEB\PYZus{}for\PYZus{}plot \PY{o}{\PYZlt{}\PYZhy{}} portfolioPricesEB \PY{o}{\PYZpc{}\PYZgt{}\PYZpc{}}
          gather\PY{p}{(}key \PY{o}{=} \PY{l+s}{\PYZdq{}}\PY{l+s}{Ticker\PYZdq{}}\PY{p}{,} value \PY{o}{=} \PY{l+s}{\PYZdq{}}\PY{l+s}{Price\PYZdq{}}\PY{p}{,} \PY{o}{\PYZhy{}}Date\PY{p}{)}
        dfEB\PYZus{}for\PYZus{}plot \PY{o}{\PYZlt{}\PYZhy{}} na.omit\PY{p}{(}dfEB\PYZus{}for\PYZus{}plot\PY{p}{)}
\end{Verbatim}


    \begin{Verbatim}[commandchars=\\\{\}]
{\color{incolor}In [{\color{incolor} }]:} \PY{o}{\PYZpc{}\PYZpc{}}R
        ggplot\PY{p}{(}dfEB\PYZus{}for\PYZus{}plot\PY{p}{,} aes\PY{p}{(}x \PY{o}{=} Date\PY{p}{,} y \PY{o}{=} Price\PY{p}{)}\PY{p}{)}\PY{o}{+}
          geom\PYZus{}line\PY{p}{(}aes\PY{p}{(}group \PY{o}{=} Ticker\PY{p}{,} linetype \PY{o}{=} Ticker\PY{p}{)}\PY{p}{)}\PY{o}{+}
          theme\PY{p}{(}panel.background \PY{o}{=}  element\PYZus{}blank\PY{p}{(}\PY{p}{)}\PY{p}{,}
                panel.grid.major \PY{o}{=} element\PYZus{}blank\PY{p}{(}\PY{p}{)}\PY{p}{,}
                panel.grid.minor \PY{o}{=} element\PYZus{}blank\PY{p}{(}\PY{p}{)}\PY{p}{)}\PY{o}{+}
          labs\PY{p}{(}title \PY{o}{=} \PY{l+s}{\PYZdq{}}\PY{l+s}{Price adjusted 2020\PYZdq{}}\PY{p}{)}
\end{Verbatim}


    \begin{center}
    \adjustimage{max size={0.9\linewidth}{0.9\paperheight}}{output_28_0.png}
    \end{center}
    { \hspace*{\fill} \\}
    
    In priciple, from the previous graphic we could conclude that our
initial conjecture was mistaken. In the previous graphic we can see huge
variations on the value of Bitcoin and very small variations on the
value of Ethereum. But the previous graphic was done with the data with
no processing, therefore the high value of the Bitcoin could be the
reason why we see very small changes on the value of Ethereum in the
graphic and no correlation between the two corruencies.

To take away the noise made by the high value of Bitcoin, we must
normalize the value of both currencies before we plot both values in the
same graphic.

    \begin{Verbatim}[commandchars=\\\{\}]
{\color{incolor}In [{\color{incolor} }]:} \PY{o}{\PYZpc{}\PYZpc{}}R
        ETHNorm \PY{o}{\PYZlt{}\PYZhy{}} na.omit\PY{p}{(}portfolioPricesEB\PY{p}{)}\PY{p}{[}\PY{l+m}{2}\PY{p}{]}\PY{o}{/}\PY{k+kp}{max}\PY{p}{(}na.omit\PY{p}{(}portfolioPricesEB\PY{p}{)}\PY{p}{[}\PY{l+m}{2}\PY{p}{]}\PY{p}{)}
        BTCNorm \PY{o}{\PYZlt{}\PYZhy{}} na.omit\PY{p}{(}portfolioPricesEB\PY{p}{)}\PY{p}{[}\PY{l+m}{3}\PY{p}{]}\PY{o}{/}\PY{k+kp}{max}\PY{p}{(}na.omit\PY{p}{(}portfolioPricesEB\PY{p}{)}\PY{p}{[}\PY{l+m}{3}\PY{p}{]}\PY{p}{)}
        NormEB \PY{o}{=} \PY{k+kt}{data.frame}\PY{p}{(}ETHNorm\PY{p}{,}BTCNorm\PY{p}{)}
\end{Verbatim}


    \begin{Verbatim}[commandchars=\\\{\}]
{\color{incolor}In [{\color{incolor} }]:} \PY{o}{\PYZpc{}\PYZpc{}}R
        NormEB \PY{o}{\PYZlt{}\PYZhy{}} rownames\PYZus{}to\PYZus{}column\PY{p}{(}NormEB\PY{p}{,} var \PY{o}{=} \PY{l+s}{\PYZdq{}}\PY{l+s}{Date\PYZdq{}}\PY{p}{)}
        dfEx\PYZus{}for\PYZus{}plot \PY{o}{\PYZlt{}\PYZhy{}} NormEB \PY{o}{\PYZpc{}\PYZgt{}\PYZpc{}}
          gather\PY{p}{(}key \PY{o}{=} \PY{l+s}{\PYZdq{}}\PY{l+s}{Ticker\PYZdq{}}\PY{p}{,} value \PY{o}{=} \PY{l+s}{\PYZdq{}}\PY{l+s}{Price\PYZdq{}}\PY{p}{,} \PY{o}{\PYZhy{}}Date\PY{p}{)}
        dfEx\PYZus{}for\PYZus{}plot \PY{o}{\PYZlt{}\PYZhy{}} na.omit\PY{p}{(}dfEx\PYZus{}for\PYZus{}plot\PY{p}{)}
\end{Verbatim}


    \begin{Verbatim}[commandchars=\\\{\}]
{\color{incolor}In [{\color{incolor} }]:} \PY{o}{\PYZpc{}\PYZpc{}}R
        ggplot\PY{p}{(}dfEx\PYZus{}for\PYZus{}plot\PY{p}{,} aes\PY{p}{(}x \PY{o}{=} Date\PY{p}{,} y \PY{o}{=} Price\PY{p}{)}\PY{p}{)}\PY{o}{+}
          geom\PYZus{}line\PY{p}{(}aes\PY{p}{(}group \PY{o}{=} Ticker\PY{p}{,} linetype \PY{o}{=} Ticker\PY{p}{)}\PY{p}{)}\PY{o}{+}
          theme\PY{p}{(}panel.background \PY{o}{=}  element\PYZus{}blank\PY{p}{(}\PY{p}{)}\PY{p}{,}
                panel.grid.major \PY{o}{=} element\PYZus{}blank\PY{p}{(}\PY{p}{)}\PY{p}{,}
                panel.grid.minor \PY{o}{=} element\PYZus{}blank\PY{p}{(}\PY{p}{)}\PY{p}{)}\PY{o}{+}
          labs\PY{p}{(}title \PY{o}{=} \PY{l+s}{\PYZdq{}}\PY{l+s}{Price normalized 2020\PYZdq{}}\PY{p}{)}
\end{Verbatim}


    \begin{center}
    \adjustimage{max size={0.9\linewidth}{0.9\paperheight}}{output_32_0.png}
    \end{center}
    { \hspace*{\fill} \\}
    
    After normalizing both values, we see that both graphics overlap in most
of the changes when the value drops (or increase) "drastically", as we
can see from the graphic above.

Finally we plot the value of Ethereum as a dependence variable of the
value of Bitcoin. With this, we try to understand the value of Ethereum
as a direct result of the value of Bitcoin. This graphic will give us a
better idea of their correlation, it worth to mention that the date is
not taken into account for this plot. Therefore, we don't see the
evolution of the values but the possible relation between them.

    \begin{Verbatim}[commandchars=\\\{\}]
{\color{incolor}In [{\color{incolor} }]:} \PY{o}{\PYZpc{}\PYZpc{}}R
        dataEB \PY{o}{\PYZlt{}\PYZhy{}} na.omit\PY{p}{(}portfolioPricesEB\PY{p}{)}
        ggplot\PY{p}{(}dataEB\PY{p}{,} aes\PY{p}{(}x \PY{o}{=} BTC.USD.Adjusted\PY{p}{,} y \PY{o}{=} ETH.USD.Adjusted\PY{p}{)}\PY{p}{)} \PY{o}{+} geom\PYZus{}point\PY{p}{(}\PY{p}{)}
\end{Verbatim}


    \begin{center}
    \adjustimage{max size={0.9\linewidth}{0.9\paperheight}}{output_34_0.png}
    \end{center}
    { \hspace*{\fill} \\}
    
    As we can see in the previous graphic, the relation we have been
observing resembles a line.

    \begin{Verbatim}[commandchars=\\\{\}]
{\color{incolor}In [{\color{incolor} }]:} \PY{o}{\PYZpc{}\PYZpc{}}R
        ggplot\PY{p}{(}dataEB\PY{p}{,} aes\PY{p}{(}x \PY{o}{=} BTC.USD.Adjusted\PY{p}{,} y \PY{o}{=} ETH.USD.Adjusted\PY{p}{)}\PY{p}{)} \PY{o}{+} geom\PYZus{}point\PY{p}{(}\PY{p}{)} \PY{o}{+}  geom\PYZus{}smooth\PY{p}{(}method \PY{o}{=} \PY{l+s}{\PYZdq{}}\PY{l+s}{lm\PYZdq{}}\PY{p}{,} se \PY{o}{=} \PY{k+kc}{TRUE}\PY{p}{,} color \PY{o}{=} \PY{l+s}{\PYZsq{}}\PY{l+s}{black\PYZsq{}}\PY{p}{)}
\end{Verbatim}


    \begin{Verbatim}[commandchars=\\\{\}]
R[write to console]: `geom\_smooth()` using formula 'y \textasciitilde{} x'


    \end{Verbatim}

    \begin{center}
    \adjustimage{max size={0.9\linewidth}{0.9\paperheight}}{output_36_1.png}
    \end{center}
    { \hspace*{\fill} \\}
    
    \subsection{Ethereum vs Basic Attention
Token}\label{ethereum-vs-basic-attention-token}

As we could see in the previous images, Ethereum and the Basic Attention
Token have no many similarities in the behaviour of their value. We can
also see that the months of February and Octber show some similarity. To
explore whether there is or not a possible correlation, we should look
at different indeces.

    \begin{Verbatim}[commandchars=\\\{\}]
{\color{incolor}In [{\color{incolor} }]:} \PY{o}{\PYZpc{}\PYZpc{}}R
        chartSeries\PY{p}{(}ETHAdjust\PY{p}{)}
        chartSeries\PY{p}{(}BATAdjust\PY{p}{)}
\end{Verbatim}


    \begin{center}
    \adjustimage{max size={0.9\linewidth}{0.9\paperheight}}{output_38_0.png}
    \end{center}
    { \hspace*{\fill} \\}
    
    \begin{center}
    \adjustimage{max size={0.9\linewidth}{0.9\paperheight}}{output_38_1.png}
    \end{center}
    { \hspace*{\fill} \\}
    
    Now lets take a look to the Moving Convergence Divergence (MACD), on top
we can find the plot of Ethereum and below is the Basic Attention Token
one.

    \begin{Verbatim}[commandchars=\\\{\}]
{\color{incolor}In [{\color{incolor} }]:} \PY{o}{\PYZpc{}\PYZpc{}}R
        \PY{l+s+sb}{`ETHUSD`} \PY{o}{\PYZpc{}\PYZgt{}\PYZpc{}} chartSeries\PY{p}{(}TA \PY{o}{=} \PY{l+s}{\PYZdq{}}\PY{l+s}{addMACD()\PYZdq{}}\PY{p}{)}
        \PY{l+s+sb}{`BATUSD`} \PY{o}{\PYZpc{}\PYZgt{}\PYZpc{}} chartSeries\PY{p}{(}TA \PY{o}{=} \PY{l+s}{\PYZdq{}}\PY{l+s}{addMACD()\PYZdq{}}\PY{p}{)}
\end{Verbatim}


    \begin{center}
    \adjustimage{max size={0.9\linewidth}{0.9\paperheight}}{output_40_0.png}
    \end{center}
    { \hspace*{\fill} \\}
    
    \begin{center}
    \adjustimage{max size={0.9\linewidth}{0.9\paperheight}}{output_40_1.png}
    \end{center}
    { \hspace*{\fill} \\}
    
    Now lets take a look to the Relative Strength Index (RSI), on top we can
find the plot of Ethereum and below is the Basic Attention Token one.

    \begin{Verbatim}[commandchars=\\\{\}]
{\color{incolor}In [{\color{incolor} }]:} \PY{o}{\PYZpc{}\PYZpc{}}R
        \PY{l+s+sb}{`ETHUSD`} \PY{o}{\PYZpc{}\PYZgt{}\PYZpc{}} chartSeries\PY{p}{(}TA \PY{o}{=} \PY{l+s}{\PYZdq{}}\PY{l+s}{addRSI()\PYZdq{}}\PY{p}{)}
        \PY{l+s+sb}{`BATUSD`} \PY{o}{\PYZpc{}\PYZgt{}\PYZpc{}} chartSeries\PY{p}{(}TA \PY{o}{=} \PY{l+s}{\PYZdq{}}\PY{l+s}{addRSI()\PYZdq{}}\PY{p}{)}
\end{Verbatim}


    \begin{center}
    \adjustimage{max size={0.9\linewidth}{0.9\paperheight}}{output_42_0.png}
    \end{center}
    { \hspace*{\fill} \\}
    
    \begin{center}
    \adjustimage{max size={0.9\linewidth}{0.9\paperheight}}{output_42_1.png}
    \end{center}
    { \hspace*{\fill} \\}
    
    Now lets take a look to the Average True Range (ATR), on top we can find
the plot of Ethereum and below is the Basic Attention Token one.

    \begin{Verbatim}[commandchars=\\\{\}]
{\color{incolor}In [{\color{incolor} }]:} \PY{o}{\PYZpc{}\PYZpc{}}R
        \PY{l+s+sb}{`ETHUSD`} \PY{o}{\PYZpc{}\PYZgt{}\PYZpc{}} chartSeries\PY{p}{(}TA \PY{o}{=} \PY{l+s}{\PYZdq{}}\PY{l+s}{addATR()\PYZdq{}}\PY{p}{)}
        \PY{l+s+sb}{`BATUSD`} \PY{o}{\PYZpc{}\PYZgt{}\PYZpc{}} chartSeries\PY{p}{(}TA \PY{o}{=} \PY{l+s}{\PYZdq{}}\PY{l+s}{addATR()\PYZdq{}}\PY{p}{)}
\end{Verbatim}


    \begin{center}
    \adjustimage{max size={0.9\linewidth}{0.9\paperheight}}{output_44_0.png}
    \end{center}
    { \hspace*{\fill} \\}
    
    \begin{center}
    \adjustimage{max size={0.9\linewidth}{0.9\paperheight}}{output_44_1.png}
    \end{center}
    { \hspace*{\fill} \\}
    
    In the previous graphic we can see that besides February and October, we
cannot see more similarities. We shoul plot the variation of adjustment
of both assets in the same graphic with respect of the date, in order to
corroborate that there is no correlation.

    \begin{Verbatim}[commandchars=\\\{\}]
{\color{incolor}In [{\color{incolor} }]:} \PY{o}{\PYZpc{}\PYZpc{}}R
        tickersEBAT \PY{o}{\PYZlt{}\PYZhy{}} \PY{k+kt}{c}\PY{p}{(}\PY{l+s}{\PYZdq{}}\PY{l+s}{BAT\PYZhy{}USD\PYZdq{}}\PY{p}{,} \PY{l+s}{\PYZdq{}}\PY{l+s}{ETH\PYZhy{}USD\PYZdq{}}\PY{p}{)}
        portfolioPricesEBAT \PY{o}{\PYZlt{}\PYZhy{}} \PY{k+kc}{NULL}
        \PY{k+kr}{for} \PY{p}{(}ticker \PY{k+kr}{in} tickersEBAT\PY{p}{)}\PY{p}{\PYZob{}}
          portfolioPricesEBAT \PY{o}{\PYZlt{}\PYZhy{}} \PY{k+kp}{cbind}\PY{p}{(}portfolioPricesEBAT\PY{p}{,} 
                                   getSymbols.yahoo\PY{p}{(}ticker\PY{p}{,}
                                                              from \PY{o}{=} starting\PYZus{}date\PY{p}{,} 
                                                              to \PY{o}{=} final\PYZus{}date\PY{p}{,}
                                                              periodicity \PY{o}{=} \PY{l+s}{\PYZdq{}}\PY{l+s}{daily\PYZdq{}}\PY{p}{,}
                                                              auto.assign \PY{o}{=} \PY{n+nb+bp}{F}\PY{p}{)}\PY{p}{[}\PY{p}{,}\PY{l+m}{6}\PY{p}{]}\PY{p}{)}
        \PY{p}{\PYZcb{}}
\end{Verbatim}


    \begin{Verbatim}[commandchars=\\\{\}]
{\color{incolor}In [{\color{incolor} }]:} \PY{o}{\PYZpc{}\PYZpc{}}R
        portfolioPricesEBAT \PY{o}{\PYZlt{}\PYZhy{}} \PY{k+kp}{as.data.frame}\PY{p}{(}portfolioPricesEBAT\PY{p}{)}
        portfolioPricesEBAT \PY{o}{\PYZlt{}\PYZhy{}} rownames\PYZus{}to\PYZus{}column\PY{p}{(}portfolioPricesEBAT\PY{p}{,} var \PY{o}{=} \PY{l+s}{\PYZdq{}}\PY{l+s}{Date\PYZdq{}}\PY{p}{)}
        dfEBAT\PYZus{}for\PYZus{}plot \PY{o}{\PYZlt{}\PYZhy{}} portfolioPricesEBAT \PY{o}{\PYZpc{}\PYZgt{}\PYZpc{}}
          gather\PY{p}{(}key \PY{o}{=} \PY{l+s}{\PYZdq{}}\PY{l+s}{Ticker\PYZdq{}}\PY{p}{,} value \PY{o}{=} \PY{l+s}{\PYZdq{}}\PY{l+s}{Price\PYZdq{}}\PY{p}{,} \PY{o}{\PYZhy{}}Date\PY{p}{)}
        dfEBAT\PYZus{}for\PYZus{}plot \PY{o}{\PYZlt{}\PYZhy{}} na.omit\PY{p}{(}dfEBAT\PYZus{}for\PYZus{}plot\PY{p}{)}
\end{Verbatim}


    \begin{Verbatim}[commandchars=\\\{\}]
{\color{incolor}In [{\color{incolor} }]:} \PY{o}{\PYZpc{}\PYZpc{}}R
        ggplot\PY{p}{(}dfEBAT\PYZus{}for\PYZus{}plot\PY{p}{,} aes\PY{p}{(}x \PY{o}{=} Date\PY{p}{,} y \PY{o}{=} Price\PY{p}{)}\PY{p}{)}\PY{o}{+}
          geom\PYZus{}line\PY{p}{(}aes\PY{p}{(}group \PY{o}{=} Ticker\PY{p}{,} linetype \PY{o}{=} Ticker\PY{p}{)}\PY{p}{)}\PY{o}{+}
          theme\PY{p}{(}panel.background \PY{o}{=}  element\PYZus{}blank\PY{p}{(}\PY{p}{)}\PY{p}{,}
                panel.grid.major \PY{o}{=} element\PYZus{}blank\PY{p}{(}\PY{p}{)}\PY{p}{,}
                panel.grid.minor \PY{o}{=} element\PYZus{}blank\PY{p}{(}\PY{p}{)}\PY{p}{)}\PY{o}{+}
          labs\PY{p}{(}title \PY{o}{=} \PY{l+s}{\PYZdq{}}\PY{l+s}{Price adjusted 2020\PYZdq{}}\PY{p}{)}
\end{Verbatim}


    \begin{center}
    \adjustimage{max size={0.9\linewidth}{0.9\paperheight}}{output_48_0.png}
    \end{center}
    { \hspace*{\fill} \\}
    
    In priciple, from the previous graphic we could conclude that our
initial conjecture is true. In the previous graphic we can see huge
variations on the value of Ethereum and no variations on the value of
the Basic Attention Token. But the previous graphic was done with the
data with no processing, therefore the high value of the Ethereum could
be the reason why we see no changes on the value of the Basic Attention
Token in the graphic and no correlation between the two corruencies.

To take away the noise made by the high value of Ethereum, we must
normalize the value of both currencies before we plot both values in the
same graphic.

    \begin{Verbatim}[commandchars=\\\{\}]
{\color{incolor}In [{\color{incolor} }]:} \PY{o}{\PYZpc{}\PYZpc{}}R
        BATNorm \PY{o}{\PYZlt{}\PYZhy{}} na.omit\PY{p}{(}portfolioPricesEBAT\PY{p}{)}\PY{p}{[}\PY{l+m}{2}\PY{p}{]}\PY{o}{/}\PY{k+kp}{max}\PY{p}{(}na.omit\PY{p}{(}portfolioPricesEBAT\PY{p}{)}\PY{p}{[}\PY{l+m}{2}\PY{p}{]}\PY{p}{)}
        ETHNorm \PY{o}{\PYZlt{}\PYZhy{}} na.omit\PY{p}{(}portfolioPricesEBAT\PY{p}{)}\PY{p}{[}\PY{l+m}{3}\PY{p}{]}\PY{o}{/}\PY{k+kp}{max}\PY{p}{(}na.omit\PY{p}{(}portfolioPricesEBAT\PY{p}{)}\PY{p}{[}\PY{l+m}{3}\PY{p}{]}\PY{p}{)}
        NormEBAT \PY{o}{=} \PY{k+kt}{data.frame}\PY{p}{(}BATNorm\PY{p}{,}ETHNorm\PY{p}{)}
\end{Verbatim}


    \begin{Verbatim}[commandchars=\\\{\}]
{\color{incolor}In [{\color{incolor} }]:} \PY{o}{\PYZpc{}\PYZpc{}}R
        NormEBAT \PY{o}{\PYZlt{}\PYZhy{}} rownames\PYZus{}to\PYZus{}column\PY{p}{(}NormEBAT\PY{p}{,} var \PY{o}{=} \PY{l+s}{\PYZdq{}}\PY{l+s}{Date\PYZdq{}}\PY{p}{)}
        dfEn\PYZus{}for\PYZus{}plot \PY{o}{\PYZlt{}\PYZhy{}} NormEBAT \PY{o}{\PYZpc{}\PYZgt{}\PYZpc{}}
          gather\PY{p}{(}key \PY{o}{=} \PY{l+s}{\PYZdq{}}\PY{l+s}{Ticker\PYZdq{}}\PY{p}{,} value \PY{o}{=} \PY{l+s}{\PYZdq{}}\PY{l+s}{Price\PYZdq{}}\PY{p}{,} \PY{o}{\PYZhy{}}Date\PY{p}{)}
        dfEn\PYZus{}for\PYZus{}plot \PY{o}{\PYZlt{}\PYZhy{}} na.omit\PY{p}{(}dfEn\PYZus{}for\PYZus{}plot\PY{p}{)}
\end{Verbatim}


    \begin{Verbatim}[commandchars=\\\{\}]
{\color{incolor}In [{\color{incolor} }]:} \PY{o}{\PYZpc{}\PYZpc{}}R
        ggplot\PY{p}{(}dfEn\PYZus{}for\PYZus{}plot\PY{p}{,} aes\PY{p}{(}x \PY{o}{=} Date\PY{p}{,} y \PY{o}{=} Price\PY{p}{)}\PY{p}{)}\PY{o}{+}
          geom\PYZus{}line\PY{p}{(}aes\PY{p}{(}group \PY{o}{=} Ticker\PY{p}{,} linetype \PY{o}{=} Ticker\PY{p}{)}\PY{p}{)}\PY{o}{+}
          theme\PY{p}{(}panel.background \PY{o}{=}  element\PYZus{}blank\PY{p}{(}\PY{p}{)}\PY{p}{,}
                panel.grid.major \PY{o}{=} element\PYZus{}blank\PY{p}{(}\PY{p}{)}\PY{p}{,}
                panel.grid.minor \PY{o}{=} element\PYZus{}blank\PY{p}{(}\PY{p}{)}\PY{p}{)}\PY{o}{+}
          labs\PY{p}{(}title \PY{o}{=} \PY{l+s}{\PYZdq{}}\PY{l+s}{Price normalized 2020\PYZdq{}}\PY{p}{)}
\end{Verbatim}


    \begin{center}
    \adjustimage{max size={0.9\linewidth}{0.9\paperheight}}{output_52_0.png}
    \end{center}
    { \hspace*{\fill} \\}
    
    After normalizing both values, we see that both graphics overlap very
rarely.

Finally we plot the value of the Basic Attention Token as a dependence
variable of the value of Ethereum. With this, we will see that our
conjecture was correct and there is not a strong correlation. It worth
to mention that the date is not taken into account for this plot.
Therefore, we don't see the evolution of the values but the possible
relation between them.

    \begin{Verbatim}[commandchars=\\\{\}]
{\color{incolor}In [{\color{incolor} }]:} \PY{o}{\PYZpc{}\PYZpc{}}R
        dataEBAT \PY{o}{\PYZlt{}\PYZhy{}} na.omit\PY{p}{(}portfolioPricesEBAT\PY{p}{)}
        ggplot\PY{p}{(}dataEBAT\PY{p}{,} aes\PY{p}{(}x \PY{o}{=} ETH.USD.Adjusted\PY{p}{,} y \PY{o}{=} BAT.USD.Adjusted\PY{p}{)}\PY{p}{)} \PY{o}{+} geom\PYZus{}point\PY{p}{(}\PY{p}{)}
\end{Verbatim}


    \begin{center}
    \adjustimage{max size={0.9\linewidth}{0.9\paperheight}}{output_54_0.png}
    \end{center}
    { \hspace*{\fill} \\}
    
    \section{The correlation between Bitcoin vs
Ethereum}\label{the-correlation-between-bitcoin-vs-ethereum}

In the previous section we explore the data of Bitcoin and Ethereum and
found that their value might be related via a linear function. In this
section we will explore more the conjecture that the value of these two
assets are related linearly. The first step will be to test the
correlation between the two variables.

    \begin{Verbatim}[commandchars=\\\{\}]
{\color{incolor}In [{\color{incolor} }]:} \PY{o}{\PYZpc{}\PYZpc{}}R
        cor.test\PY{p}{(}dataEB\PY{p}{[}\PY{p}{,}\PY{l+m}{3}\PY{p}{]}\PY{p}{,} dataEB\PY{p}{[}\PY{p}{,}\PY{l+m}{2}\PY{p}{]}\PY{p}{)}
\end{Verbatim}


    \begin{Verbatim}[commandchars=\\\{\}]

	Pearson's product-moment correlation

data:  dataEB[, 3] and dataEB[, 2]
t = 52.925, df = 360, p-value < 2.2e-16
alternative hypothesis: true correlation is not equal to 0
95 percent confidence interval:
 0.9283361 0.9520370
sample estimates:
      cor 
0.9413364 


    \end{Verbatim}

    As we can see that the p-value is very low. Therefore we should do a
linear regression to build a model and test the model with new data,
this in order to understand if the behaviour exposed by Bitcoin and
Ethereum during 2020 was a singularity of that time interval, or if the
value of both assets are linearly related in genral via the market cap
and traders.

    \subsection{Buulding and testing the
model}\label{buulding-and-testing-the-model}

Even though we could build the model with all the data of 2020 and test
it with the data of 2021, we will divide the data in a training set and
test set for good practices and to not build a "perfect model" that is
not good to make predictions.

    \begin{Verbatim}[commandchars=\\\{\}]
{\color{incolor}In [{\color{incolor} }]:} \PY{o}{\PYZpc{}\PYZpc{}}R
        split \PY{o}{=} sample.split\PY{p}{(}dataEB\PY{o}{\PYZdl{}}ETH.USD.Adjusted\PY{p}{,} SplitRatio \PY{o}{=} \PY{l+m}{0.8}\PY{p}{)}
        training\PYZus{}set \PY{o}{=} \PY{k+kp}{subset}\PY{p}{(}dataEB\PY{p}{,} split \PY{o}{==} \PY{k+kc}{TRUE}\PY{p}{)}
        test\PYZus{}set \PY{o}{=} \PY{k+kp}{subset}\PY{p}{(}dataEB\PY{p}{,} split \PY{o}{==} \PY{k+kc}{FALSE}\PY{p}{)}
\end{Verbatim}


    \begin{Verbatim}[commandchars=\\\{\}]
{\color{incolor}In [{\color{incolor} }]:} \PY{o}{\PYZpc{}\PYZpc{}}R
        regressor \PY{o}{=} lm\PY{p}{(}formula \PY{o}{=} ETH.USD.Adjusted \PY{o}{\PYZti{}} BTC.USD.Adjusted\PY{p}{,} 
                       data \PY{o}{=} training\PYZus{}set\PY{p}{)}
\end{Verbatim}


    Now that we have build the model, we can plot the test set of 2020 (the
points in red) against the prediction of our model (the blue line). We
can also compare our model with our initial overview of the correlation.

    \begin{Verbatim}[commandchars=\\\{\}]
{\color{incolor}In [{\color{incolor} }]:} \PY{o}{\PYZpc{}\PYZpc{}}R
        ggplot\PY{p}{(}\PY{p}{)} \PY{o}{+}
          geom\PYZus{}point\PY{p}{(}aes\PY{p}{(}x \PY{o}{=} test\PYZus{}set\PY{o}{\PYZdl{}}BTC.USD.Adjusted\PY{p}{,} y \PY{o}{=} test\PYZus{}set\PY{o}{\PYZdl{}}ETH.USD.Adjusted\PY{p}{)}\PY{p}{,} 
                     colour \PY{o}{=} \PY{l+s}{\PYZsq{}}\PY{l+s}{red\PYZsq{}}\PY{p}{)} \PY{o}{+}
          geom\PYZus{}line\PY{p}{(}aes\PY{p}{(}x \PY{o}{=} training\PYZus{}set\PY{o}{\PYZdl{}}BTC.USD.Adjusted\PY{p}{,} y \PY{o}{=} predict\PY{p}{(}regressor\PY{p}{,} newdata \PY{o}{=} training\PYZus{}set\PY{p}{)}\PY{p}{)}\PY{p}{,}
                    colour \PY{o}{=} \PY{l+s}{\PYZsq{}}\PY{l+s}{blue\PYZsq{}}\PY{p}{)} \PY{o}{+}
          xlab\PY{p}{(}\PY{l+s}{\PYZsq{}}\PY{l+s}{BTC\PYZsq{}}\PY{p}{)} \PY{o}{+}
          ylab\PY{p}{(}\PY{l+s}{\PYZsq{}}\PY{l+s}{ETH\PYZsq{}}\PY{p}{)}
\end{Verbatim}


    \begin{center}
    \adjustimage{max size={0.9\linewidth}{0.9\paperheight}}{output_62_0.png}
    \end{center}
    { \hspace*{\fill} \\}
    
    \begin{Verbatim}[commandchars=\\\{\}]
{\color{incolor}In [{\color{incolor} }]:} \PY{o}{\PYZpc{}\PYZpc{}}R
        ggplot\PY{p}{(}dataEB\PY{p}{,} aes\PY{p}{(}x \PY{o}{=} BTC.USD.Adjusted\PY{p}{,} y \PY{o}{=} ETH.USD.Adjusted\PY{p}{)}\PY{p}{)} \PY{o}{+} geom\PYZus{}point\PY{p}{(}\PY{p}{)} \PY{o}{+}  geom\PYZus{}smooth\PY{p}{(}method \PY{o}{=} \PY{l+s}{\PYZdq{}}\PY{l+s}{lm\PYZdq{}}\PY{p}{,} se \PY{o}{=} \PY{k+kc}{TRUE}\PY{p}{,} color \PY{o}{=} \PY{l+s}{\PYZsq{}}\PY{l+s}{black\PYZsq{}}\PY{p}{)}
\end{Verbatim}


    \begin{Verbatim}[commandchars=\\\{\}]
R[write to console]: `geom\_smooth()` using formula 'y \textasciitilde{} x'


    \end{Verbatim}

    \begin{center}
    \adjustimage{max size={0.9\linewidth}{0.9\paperheight}}{output_63_1.png}
    \end{center}
    { \hspace*{\fill} \\}
    
    \subsection{Predicting new data (January
2021)}\label{predicting-new-data-january-2021}

Now that we have build our model and seen that it fits with our initial
conjecture, we can proceed to test the model with the data of January
2021. With this test we can conclude whether the correlation observed in
2020 trend to continue during 2021, or if the recent moves in the market
has modify this relation. We will start by downloading the data and
showiong the Japanese Candlesticks before we apply our model, this
should gives us an initial intuition about how well our model behaves
with the new data.

    \begin{Verbatim}[commandchars=\\\{\}]
{\color{incolor}In [{\color{incolor} }]:} \PY{o}{\PYZpc{}\PYZpc{}}R
        starting\PYZus{}prediction\PYZus{}date \PY{o}{\PYZlt{}\PYZhy{}} \PY{l+s}{\PYZdq{}}\PY{l+s}{2021\PYZhy{}1\PYZhy{}1\PYZdq{}}
        final\PYZus{}prediction\PYZus{}date \PY{o}{\PYZlt{}\PYZhy{}} \PY{l+s}{\PYZdq{}}\PY{l+s}{2021\PYZhy{}1\PYZhy{}31\PYZdq{}}
\end{Verbatim}


    \begin{Verbatim}[commandchars=\\\{\}]
{\color{incolor}In [{\color{incolor} }]:} \PY{o}{\PYZpc{}\PYZpc{}}R
        BTCUSD\PYZus{}for\PYZus{}prediction \PY{o}{\PYZlt{}\PYZhy{}} getSymbols.yahoo\PY{p}{(}\PY{l+s}{\PYZdq{}}\PY{l+s}{BTC\PYZhy{}USD\PYZdq{}}\PY{p}{,} from \PY{o}{=} starting\PYZus{}prediction\PYZus{}date\PY{p}{,} to \PY{o}{=} final\PYZus{}prediction\PYZus{}date\PY{p}{,} auto.assign \PY{o}{=} \PY{n+nb+bp}{F}\PY{p}{)}\PY{p}{[}\PY{p}{,}\PY{p}{]}
        BTCAdjust\PYZus{}for\PYZus{}prediction \PY{o}{\PYZlt{}\PYZhy{}} BTCUSD\PYZus{}for\PYZus{}prediction\PY{p}{[}\PY{p}{,}\PY{l+m}{6}\PY{p}{]}
        chartSeries\PY{p}{(}BTCUSD\PYZus{}for\PYZus{}prediction\PY{p}{)}
\end{Verbatim}


    \begin{center}
    \adjustimage{max size={0.9\linewidth}{0.9\paperheight}}{output_66_0.png}
    \end{center}
    { \hspace*{\fill} \\}
    
    \begin{Verbatim}[commandchars=\\\{\}]
{\color{incolor}In [{\color{incolor} }]:} \PY{o}{\PYZpc{}\PYZpc{}}R
        ETHUSD\PYZus{}for\PYZus{}prediction \PY{o}{\PYZlt{}\PYZhy{}} getSymbols.yahoo\PY{p}{(}\PY{l+s}{\PYZdq{}}\PY{l+s}{ETH\PYZhy{}USD\PYZdq{}}\PY{p}{,} from \PY{o}{=} starting\PYZus{}prediction\PYZus{}date\PY{p}{,} to \PY{o}{=} final\PYZus{}prediction\PYZus{}date\PY{p}{,} auto.assign \PY{o}{=} \PY{n+nb+bp}{F}\PY{p}{)}\PY{p}{[}\PY{p}{,}\PY{p}{]}
        ETHAdjust\PYZus{}for\PYZus{}prediction \PY{o}{\PYZlt{}\PYZhy{}} ETHUSD\PYZus{}for\PYZus{}prediction\PY{p}{[}\PY{p}{,}\PY{l+m}{6}\PY{p}{]}
        chartSeries\PY{p}{(}ETHUSD\PYZus{}for\PYZus{}prediction\PY{p}{)}
\end{Verbatim}


    \begin{center}
    \adjustimage{max size={0.9\linewidth}{0.9\paperheight}}{output_67_0.png}
    \end{center}
    { \hspace*{\fill} \\}
    
    From the two previous graphics we could conclude that the relation
between both assets during 2021 will not follow the trend of 2020. We
can observe that Bitcoin has lost a lot of value in the last two thirds
of January, meanwhile Ethereum has had a constant growing trend during
January. Even though it lost a lot of value during some days, Ethereum
recovers the value very quick.

Let us prepare the data and test our model with the new data, being the
data point the red ones and the model prediction the blue line.

    \begin{Verbatim}[commandchars=\\\{\}]
{\color{incolor}In [{\color{incolor} }]:} \PY{o}{\PYZpc{}\PYZpc{}}R
        EB\PYZus{}for\PYZus{}prediction \PY{o}{\PYZlt{}\PYZhy{}} \PY{k+kc}{NULL}
        \PY{k+kr}{for} \PY{p}{(}ticker \PY{k+kr}{in} tickersEB\PY{p}{)}\PY{p}{\PYZob{}}
          EB\PYZus{}for\PYZus{}prediction \PY{o}{\PYZlt{}\PYZhy{}} \PY{k+kp}{cbind}\PY{p}{(}EB\PYZus{}for\PYZus{}prediction\PY{p}{,} 
                                   getSymbols.yahoo\PY{p}{(}ticker\PY{p}{,}
                                                              from \PY{o}{=} starting\PYZus{}prediction\PYZus{}date\PY{p}{,} 
                                                              to \PY{o}{=} final\PYZus{}prediction\PYZus{}date\PY{p}{,}
                                                              periodicity \PY{o}{=} \PY{l+s}{\PYZdq{}}\PY{l+s}{daily\PYZdq{}}\PY{p}{,}
                                                              auto.assign \PY{o}{=} \PY{n+nb+bp}{F}\PY{p}{)}\PY{p}{[}\PY{p}{,}\PY{l+m}{6}\PY{p}{]}\PY{p}{)}
          
        \PY{p}{\PYZcb{}}
\end{Verbatim}


    \begin{Verbatim}[commandchars=\\\{\}]
{\color{incolor}In [{\color{incolor} }]:} \PY{o}{\PYZpc{}\PYZpc{}}R
        ETH\PYZus{}pred \PY{o}{=} \PY{k+kt}{data.frame}\PY{p}{(}EB\PYZus{}for\PYZus{}prediction\PY{o}{\PYZdl{}}BTC.USD.Adjusted\PY{p}{,} predict\PY{p}{(}regressor\PY{p}{,} newdata \PY{o}{=} EB\PYZus{}for\PYZus{}prediction\PY{p}{)}\PY{p}{)}
        \PY{k+kp}{names}\PY{p}{(}ETH\PYZus{}pred\PY{p}{)}\PY{p}{[}\PY{l+m}{2}\PY{p}{]} \PY{o}{\PYZlt{}\PYZhy{}} \PY{l+s}{\PYZdq{}}\PY{l+s}{ETH.USD.Adjusted\PYZdq{}}
        ETH\PYZus{}pred\PY{o}{\PYZlt{}\PYZhy{}} ETH\PYZus{}pred\PY{p}{[}\PY{p}{,} \PY{k+kt}{c}\PY{p}{(}\PY{l+m}{2}\PY{p}{,}\PY{l+m}{1}\PY{p}{)}\PY{p}{]}
\end{Verbatim}


    \begin{Verbatim}[commandchars=\\\{\}]
{\color{incolor}In [{\color{incolor} }]:} \PY{o}{\PYZpc{}\PYZpc{}}R
        ggplot\PY{p}{(}\PY{p}{)} \PY{o}{+}
               geom\PYZus{}line\PY{p}{(}aes\PY{p}{(}x \PY{o}{=} ETH\PYZus{}pred\PY{o}{\PYZdl{}}BTC.USD.Adjusted\PY{p}{,} y \PY{o}{=} ETH\PYZus{}pred\PY{o}{\PYZdl{}}ETH.USD.Adjusted\PY{p}{)} \PY{p}{,} colour \PY{o}{=} \PY{l+s}{\PYZsq{}}\PY{l+s}{blue\PYZsq{}}\PY{p}{)} \PY{o}{+}
               geom\PYZus{}point\PY{p}{(}aes\PY{p}{(}x \PY{o}{=} EB\PYZus{}for\PYZus{}prediction\PY{o}{\PYZdl{}}BTC.USD.Adjusted\PY{p}{,} y \PY{o}{=} EB\PYZus{}for\PYZus{}prediction\PY{o}{\PYZdl{}}ETH.USD.Adjusted\PY{p}{)} \PY{p}{,} colour \PY{o}{=} \PY{l+s}{\PYZsq{}}\PY{l+s}{red\PYZsq{}}\PY{p}{)} \PY{o}{+}
               xlab\PY{p}{(}\PY{l+s}{\PYZsq{}}\PY{l+s}{BTC\PYZsq{}}\PY{p}{)} \PY{o}{+}
               ylab\PY{p}{(}\PY{l+s}{\PYZsq{}}\PY{l+s}{ETH\PYZsq{}}\PY{p}{)}
\end{Verbatim}


    \begin{center}
    \adjustimage{max size={0.9\linewidth}{0.9\paperheight}}{output_71_0.png}
    \end{center}
    { \hspace*{\fill} \\}
    
    As we can see in the previous graphic, our model fails to predict the
relation between this two assets observed during 2021.

    \subsection{The big picture}\label{the-big-picture}

Before we make conclusion we have to get the big picture. We know that
this is a small set of data and the error observed in this picture may
be less than the error produced when the model was constructed with the
2020 data.

    \begin{Verbatim}[commandchars=\\\{\}]
{\color{incolor}In [{\color{incolor} }]:} \PY{o}{\PYZpc{}\PYZpc{}}R
        EB\PYZus{}whole \PY{o}{\PYZlt{}\PYZhy{}} \PY{k+kc}{NULL}
        \PY{k+kr}{for} \PY{p}{(}ticker \PY{k+kr}{in} tickersEB\PY{p}{)}\PY{p}{\PYZob{}}
          EB\PYZus{}whole \PY{o}{\PYZlt{}\PYZhy{}} \PY{k+kp}{cbind}\PY{p}{(}EB\PYZus{}whole\PY{p}{,} 
                                   getSymbols.yahoo\PY{p}{(}ticker\PY{p}{,}
                                                              from \PY{o}{=} starting\PYZus{}date\PY{p}{,} 
                                                              to \PY{o}{=} final\PYZus{}prediction\PYZus{}date\PY{p}{,}
                                                              periodicity \PY{o}{=} \PY{l+s}{\PYZdq{}}\PY{l+s}{daily\PYZdq{}}\PY{p}{,}
                                                              auto.assign \PY{o}{=} \PY{n+nb+bp}{F}\PY{p}{)}\PY{p}{[}\PY{p}{,}\PY{l+m}{6}\PY{p}{]}\PY{p}{)}
          
        \PY{p}{\PYZcb{}}
\end{Verbatim}


    \begin{Verbatim}[commandchars=\\\{\}]
{\color{incolor}In [{\color{incolor} }]:} \PY{o}{\PYZpc{}\PYZpc{}}R
        ETH\PYZus{}pred\PYZus{}whole \PY{o}{=} \PY{k+kt}{data.frame}\PY{p}{(}EB\PYZus{}whole\PY{o}{\PYZdl{}}BTC.USD.Adjusted\PY{p}{,} predict\PY{p}{(}regressor\PY{p}{,} newdata \PY{o}{=} EB\PYZus{}whole\PY{p}{)}\PY{p}{)}
        \PY{k+kp}{names}\PY{p}{(}ETH\PYZus{}pred\PYZus{}whole\PY{p}{)}\PY{p}{[}\PY{l+m}{2}\PY{p}{]} \PY{o}{\PYZlt{}\PYZhy{}} \PY{l+s}{\PYZdq{}}\PY{l+s}{ETH.USD.Adjusted\PYZdq{}}
        ETH\PYZus{}pred\PYZus{}whole\PY{o}{\PYZlt{}\PYZhy{}} ETH\PYZus{}pred\PYZus{}whole\PY{p}{[}\PY{p}{,} \PY{k+kt}{c}\PY{p}{(}\PY{l+m}{2}\PY{p}{,}\PY{l+m}{1}\PY{p}{)}\PY{p}{]}
\end{Verbatim}


    \begin{Verbatim}[commandchars=\\\{\}]
{\color{incolor}In [{\color{incolor} }]:} \PY{o}{\PYZpc{}\PYZpc{}}R
        ggplot\PY{p}{(}\PY{p}{)} \PY{o}{+}
               geom\PYZus{}line\PY{p}{(}aes\PY{p}{(}x \PY{o}{=} ETH\PYZus{}pred\PYZus{}whole\PY{o}{\PYZdl{}}BTC.USD.Adjusted\PY{p}{,} y \PY{o}{=} ETH\PYZus{}pred\PYZus{}whole\PY{o}{\PYZdl{}}ETH.USD.Adjusted\PY{p}{)} \PY{p}{,} colour \PY{o}{=} \PY{l+s}{\PYZsq{}}\PY{l+s}{blue\PYZsq{}}\PY{p}{)} \PY{o}{+}
               geom\PYZus{}point\PY{p}{(}aes\PY{p}{(}x \PY{o}{=} EB\PYZus{}whole\PY{o}{\PYZdl{}}BTC.USD.Adjusted\PY{p}{,} y \PY{o}{=} EB\PYZus{}whole\PY{o}{\PYZdl{}}ETH.USD.Adjusted\PY{p}{)} \PY{p}{,} colour \PY{o}{=} \PY{l+s}{\PYZsq{}}\PY{l+s}{red\PYZsq{}}\PY{p}{)} \PY{o}{+}
               xlab\PY{p}{(}\PY{l+s}{\PYZsq{}}\PY{l+s}{BTC\PYZsq{}}\PY{p}{)} \PY{o}{+}
               ylab\PY{p}{(}\PY{l+s}{\PYZsq{}}\PY{l+s}{ETH\PYZsq{}}\PY{p}{)}
\end{Verbatim}


    \begin{center}
    \adjustimage{max size={0.9\linewidth}{0.9\paperheight}}{output_76_0.png}
    \end{center}
    { \hspace*{\fill} \\}
    
    As we can observe in the previous graphic in which we used the data from
2020 and the one from January 2021, the linear relation that we observed
during 2020 does not hold any longer with the new data.

    \subsection{The Volume}\label{the-volume}

Let us finish the section by taking a look to the volume of of Bitcoin
and Ethereum. We have seen that during 2020 the value of both assets
were related, now the question is whether the volume are also related.

    \begin{Verbatim}[commandchars=\\\{\}]
{\color{incolor}In [{\color{incolor} }]:} \PY{o}{\PYZpc{}\PYZpc{}}R
        BTCVolume \PY{o}{\PYZlt{}\PYZhy{}} BTCUSD\PY{p}{[}\PY{p}{,}\PY{l+m}{5}\PY{p}{]}
        chartSeries\PY{p}{(}BTCVolume\PY{p}{)}
\end{Verbatim}


    \begin{center}
    \adjustimage{max size={0.9\linewidth}{0.9\paperheight}}{output_79_0.png}
    \end{center}
    { \hspace*{\fill} \\}
    
    \begin{Verbatim}[commandchars=\\\{\}]
{\color{incolor}In [{\color{incolor} }]:} \PY{o}{\PYZpc{}\PYZpc{}}R
        ETHVolume \PY{o}{\PYZlt{}\PYZhy{}} ETHUSD\PY{p}{[}\PY{p}{,}\PY{l+m}{5}\PY{p}{]}
        chartSeries\PY{p}{(}ETHVolume\PY{p}{)}
\end{Verbatim}


    \begin{center}
    \adjustimage{max size={0.9\linewidth}{0.9\paperheight}}{output_80_0.png}
    \end{center}
    { \hspace*{\fill} \\}
    
    We can see in the graphics above that the volumes lokk very similar.

    \begin{Verbatim}[commandchars=\\\{\}]
{\color{incolor}In [{\color{incolor} }]:} \PY{o}{\PYZpc{}\PYZpc{}}R
        EB\PYZus{}Volume \PY{o}{\PYZlt{}\PYZhy{}} \PY{k+kc}{NULL}
        \PY{k+kr}{for} \PY{p}{(}ticker \PY{k+kr}{in} tickersEB\PY{p}{)}\PY{p}{\PYZob{}}
          EB\PYZus{}Volume \PY{o}{\PYZlt{}\PYZhy{}} \PY{k+kp}{cbind}\PY{p}{(}EB\PYZus{}Volume\PY{p}{,} 
                                   getSymbols.yahoo\PY{p}{(}ticker\PY{p}{,}
                                                              from \PY{o}{=} starting\PYZus{}date\PY{p}{,} 
                                                              to \PY{o}{=} final\PYZus{}date\PY{p}{,}
                                                              periodicity \PY{o}{=} \PY{l+s}{\PYZdq{}}\PY{l+s}{daily\PYZdq{}}\PY{p}{,}
                                                              auto.assign \PY{o}{=} \PY{n+nb+bp}{F}\PY{p}{)}\PY{p}{[}\PY{p}{,}\PY{l+m}{5}\PY{p}{]}\PY{p}{)}
          
        \PY{p}{\PYZcb{}}
\end{Verbatim}


    \begin{Verbatim}[commandchars=\\\{\}]
{\color{incolor}In [{\color{incolor} }]:} \PY{o}{\PYZpc{}\PYZpc{}}R
        EB\PYZus{}Volume \PY{o}{\PYZlt{}\PYZhy{}} na.omit\PY{p}{(}EB\PYZus{}Volume\PY{p}{)}
        ggplot\PY{p}{(}EB\PYZus{}Volume\PY{p}{,} aes\PY{p}{(}x \PY{o}{=} BTC.USD.Volume\PY{p}{,} y \PY{o}{=} ETH.USD.Volume\PY{p}{)}\PY{p}{)} \PY{o}{+} geom\PYZus{}point\PY{p}{(}colour \PY{o}{=} \PY{l+s}{\PYZsq{}}\PY{l+s}{red\PYZsq{}}\PY{p}{)}
\end{Verbatim}


    \begin{center}
    \adjustimage{max size={0.9\linewidth}{0.9\paperheight}}{output_83_0.png}
    \end{center}
    { \hspace*{\fill} \\}
    
    From the previous picture we can conclude that the volumes are related
but the trend is not clearly linear as in the case of the values.

    \section{The correlation between Ethereum and the Basic Attention
Token}\label{the-correlation-between-ethereum-and-the-basic-attention-token}

As we saw in the third section, the value of Ethereum and the value of
the Basic Attention Token have no correlation. In this section we will
take a look to the volume of these two assets, this would give us an
overview of the popularity of both assets.

    \begin{Verbatim}[commandchars=\\\{\}]
{\color{incolor}In [{\color{incolor} }]:} \PY{o}{\PYZpc{}\PYZpc{}}R
        ETHVolume \PY{o}{\PYZlt{}\PYZhy{}} ETHUSD\PY{p}{[}\PY{p}{,}\PY{l+m}{5}\PY{p}{]}
        chartSeries\PY{p}{(}ETHVolume\PY{p}{)}
\end{Verbatim}


    \begin{center}
    \adjustimage{max size={0.9\linewidth}{0.9\paperheight}}{output_86_0.png}
    \end{center}
    { \hspace*{\fill} \\}
    
    \begin{Verbatim}[commandchars=\\\{\}]
{\color{incolor}In [{\color{incolor} }]:} \PY{o}{\PYZpc{}\PYZpc{}}R
        BATVolume \PY{o}{\PYZlt{}\PYZhy{}} BATUSD\PY{p}{[}\PY{p}{,}\PY{l+m}{5}\PY{p}{]}
        chartSeries\PY{p}{(}BATVolume\PY{p}{)}
\end{Verbatim}


    \begin{center}
    \adjustimage{max size={0.9\linewidth}{0.9\paperheight}}{output_87_0.png}
    \end{center}
    { \hspace*{\fill} \\}
    
    From the previous images it is not clear if there is a corelation
between these two volumes, in the first part of 2020 the volumes are
clearly not corelated, but in the second part the volumes look similar.
To get a better understanding we should plot the volume of the Basic
Attention Token as a dependent variable of the volume of Ethereum.

    \begin{Verbatim}[commandchars=\\\{\}]
{\color{incolor}In [{\color{incolor} }]:} \PY{o}{\PYZpc{}\PYZpc{}}R
        EBAT\PYZus{}Volume \PY{o}{\PYZlt{}\PYZhy{}} \PY{k+kc}{NULL}
        \PY{k+kr}{for} \PY{p}{(}ticker \PY{k+kr}{in} tickersEBAT\PY{p}{)}\PY{p}{\PYZob{}}
          EBAT\PYZus{}Volume \PY{o}{\PYZlt{}\PYZhy{}} \PY{k+kp}{cbind}\PY{p}{(}EBAT\PYZus{}Volume\PY{p}{,} 
                                   getSymbols.yahoo\PY{p}{(}ticker\PY{p}{,}
                                                              from \PY{o}{=} starting\PYZus{}date\PY{p}{,} 
                                                              to \PY{o}{=} final\PYZus{}date\PY{p}{,}
                                                              periodicity \PY{o}{=} \PY{l+s}{\PYZdq{}}\PY{l+s}{daily\PYZdq{}}\PY{p}{,}
                                                              auto.assign \PY{o}{=} \PY{n+nb+bp}{F}\PY{p}{)}\PY{p}{[}\PY{p}{,}\PY{l+m}{5}\PY{p}{]}\PY{p}{)}
          
        \PY{p}{\PYZcb{}}
\end{Verbatim}


    \begin{Verbatim}[commandchars=\\\{\}]
{\color{incolor}In [{\color{incolor} }]:} \PY{o}{\PYZpc{}\PYZpc{}}R
        EBAT\PYZus{}Volume \PY{o}{\PYZlt{}\PYZhy{}} na.omit\PY{p}{(}EBAT\PYZus{}Volume\PY{p}{)}
        ggplot\PY{p}{(}EBAT\PYZus{}Volume\PY{p}{,} aes\PY{p}{(}x \PY{o}{=} ETH.USD.Volume\PY{p}{,} y \PY{o}{=} BAT.USD.Volume\PY{p}{)}\PY{p}{)} \PY{o}{+} geom\PYZus{}point\PY{p}{(}colour \PY{o}{=} \PY{l+s}{\PYZsq{}}\PY{l+s}{red\PYZsq{}}\PY{p}{)}
\end{Verbatim}


    \begin{center}
    \adjustimage{max size={0.9\linewidth}{0.9\paperheight}}{output_90_0.png}
    \end{center}
    { \hspace*{\fill} \\}
    
    From the last image we can see that an increase in the volume of
Ethereum does not implies an increase on the value of the Basic
Attention Token, this makes us think that these two assets are not
related in value and neither in volume.

    \section{Conclusions}\label{conclusions}

After making this short study we can conclude that there is a relation
between Bitcoin and Ethereum, but it looks that this relation is going
to change during 2021. This could be due to many reasons and it deserves
further studies. In contrast of the simple idea that a token based on
the Ethereum blockchain would have a tight relation to Ethereum, we have
found that this is not neccesarly true. We have show that the Basic
Attention Token and Ethereum don't have a tight relation neither in
value nor in volume. Clearly these are not the only cryptoassets, there
are many other such as Binance coin, Lite coin, etc. This short study
can be made with any other cryptoassets to understand whether their are
related via the market cap.

Finally, there is not a general correlation between any two
cryptoassets, we showed two cases in which the assets are corelated and
another one in which the assets are not.

\section{Author information}

Miguel Moreno

Github: Miguelwan

Webpage: miguelmath.com

Email: contact@miguelmath.com


    % Add a bibliography block to the postdoc
    
    
    
    \end{document}
